% pdfLaTeX document; UTF-8
\documentclass[a4paper]{article}
\usepackage[utf8]{inputenc}
\usepackage[T1]{fontenc}
\usepackage{lmodern,textcomp}
\usepackage{geometry}
\usepackage{xcolor}
\usepackage[unicode,colorlinks]{hyperref}
\hypersetup{linkcolor=blue!75!black,urlcolor=green!45!black}
\usepackage{shortvrb}
\MakeShortVerb{\|}
\usepackage{verbatim}
\newenvironment{myverbatim}
  {\begin{quote}\small\verbatim}
  {\endverbatim\end{quote}}
\newcommand{\PkgVersion}{0.4}
\newcommand{\PkgDate}{2016/11/11}
\newcommand{\Pkg}[1]{\textsf{#1}}
\newcommand{\Meta}[1]{$\langle$\textit{#1}$\rangle$}
\newcommand{\Note}{\par\noindent \emph{Note:}\quad}
\newcommand{\Means}{:\hspace{1em plus 1em}}
\newcommand{\wbr}{\linebreak[0]}
\providecommand{\pTeX}{p\TeX}
\providecommand{\pLaTeX}{p\LaTeX}
\providecommand{\pdfTeX}{pdf\/\TeX}
%-----------------------------------------------------------
\begin{document}
\title{The \Pkg{bxcjkjatype} Package}
\author{Takayuki YATO (aka.~``ZR'')}
\date{v\PkgVersion\quad[\PkgDate]}
\maketitle

\begin{abstract}
This package provides working configuration of the \Pkg{CJK} package
suitable for Japanese typesetting of moderate quality.
Moreover, it facilitates use of the \Pkg{CJK} package for {\pLaTeX}
users, by providing commands that are similar to those used by the
{\pLaTeX} kernel and some other packages used with it.
\end{abstract}

\tableofcontents

%===========================================================
\section{Package Loading}
\label{sec:loading}

\begin{verbatim}
\usepackage[<option>,...]{bxcjkjatype}
\end{verbatim}

The available options are described hereafter.

%-------------------
\subsection{Options for auto-wrapping}

These options enable one to wrap the document body with a |CJK(*)|
environment automatically and safely.
They are suitable when a document contains much amount of CJK text,
or some ``moving arguments'' hold CJK text.

\begin{itemize}
\item |whole|, |wholeCJK*|\Means
  Wraps the whole document body with a |CJK*| environment
  (precisely speaking, with |\begin{uCJK*}|\ldots\wbr|\end{uCJK*}|).
\item |wholeCJK|\Means
  Wraps the whole document body with a |CJK| environment
  (precisely speaking, with |\begin{uCJK}|\ldots\wbr|\end{uCJK}|).
\item |nowhole| (default)\Means
  Negation of |wholeCJK*| or |wholeCJK|.
\end{itemize}

%-------------------
\subsection{Options for ``auto-tilde''}

The option |autotilde| triggers automatic invocation of |\CJKtilde|,
which makes a tilde character (|~|) insert ``shibuaki'' (a thin space
between alphabetic and ideographic letters) rather than a no-break space
(standard).
No-break spaces can still be inserted by the command |\nbs|, and
|\standardtilde| cancels the effect of |\CJKtilde|.
(The commands mentioned here belong to \Pkg{CJK} package.)

\begin{itemize}
\item |autotilde|\Means
  Makes |\CJKtilde| invoked at the beginning of every |CJK(*)|
  environment.
\item |noautotilde| (default)\Means
  Negation of |autotilde|.
\end{itemize}

%-------------------
\subsection{Options for font-mapping}

One can use preset font mappings in the same way as in the
\href{http://www.ctan.org/pkg/pxchfon}{\Pkg{pxchfon} package}.
Please refer to the manual of that package for detailed explanation
of this feature.

\begin{itemize}
\item |oneweight|, |nooneweight|\Means
  The same as in \Pkg{pxchfon}.
\item One can use font preset options (such as |ms|) which are available
  in \Pkg{pxchfon} (except obsolete ones).
\item |ttfname=|\Meta{pattern}\Means
  Specifies the pattern of the TTF font names which are used when TTC
  substitution is employed.
  For example, when the option |ttfname=*_1| is given, the font
  ``index 0 of mogam.ttc'' will map to ``mogam\_1.ttf'', and similarly,
  ``index 1'' to ``mogam\_2.ttf'' and so on.
\item |ipaex-type1|\Means
  Disables the font management of this package and directly uses the
  families provided by the ipaex-type1 package, namely |ipxm| and |ipxg|.
  In this setting the value of |\mcdefault| is |ipxm| and the value of
  |\gtdefault| and |\mgdefault| is |ipxg|, so that the higher level
  commands (such as |\sffamily| and |\gtfamily|) can work correctly.
\end{itemize}

%-------------------
\subsection{Options for CJK font scaling}

\begin{itemize}
\item |scale=|\Meta{real}\Means
  Sets the scaling factor for CJK fonts.
\end{itemize}

(When using version 0.3 or later, one can employ the scaling even with the
|ipaex-type1| option.)

%-------------------
\subsection{Options for configuring ``Shibuaki'' in PDF strings}

In {\LaTeX} grammar |~| represents a non-breaking space.
Accordingly, when the \Pkg{hyperref} package generates PDF strings,
|~| in {\LaTeX} text will be converted to a space character.

However, when |\CJKtilde| of the \Pkg{CJK} package is effective
the meaning of |~| changes to ``shibuaki''.
The shibuaki is device on typesetting and is not a space as text data.
Thus when this package is loaded, |~| with |\CJKtilde| effective
is tailored to be deleted in conversion to PDF strings.

Moreover this behavior can be configured by options.

\begin{itemize}
\item |noCJKtildeasspace| (default)\Means
  When |\CJKtilde| is effective, |~| will be deleted.
  in conversion to PDF strings.
\item |CJKtildeasspace|\Means
  When |\CJKtilde| is effective, |~| will be converted
  to a space character.
  (This is the same behavior as when this package is not used.)
\end{itemize}

%-------------------
\subsection{Other options}

\begin{itemize}
\item |everypage|\Means
  Outputs the font mapping information on every page of the resulted
  DVI document.
  Available only with |dvipdfmx| driver.
\item |noeverypage| (default)\Means
  Negation of |everypage|.
\item driver options\Means
  |pdftex|, |dvipdfmx|, |dvips| and |none| are available.
  The driver setting is relevant only when using font mappings other
  than the default one (ipaex-type1 fonts), so one need not care of
  drivers in using default fonts.
  Moreover, non-default font mappings are supported only by |pdftex|
  and |dvipdfmx|, and these two values are auto-detected
  (|pdftex| is default in PDF mode and |dvipdfmx| in DVI mode).
  Thus one will never need to specify the driver.
\item |substmingoth|\Means
  Applies the substituion of families |min|, |goth| and |maru|
  (used conventionally for Japanese) with families |mc|, |gt| and |mg|
  (standard in this package).
\item |nosubstmingoth| (default)\Means
  Negation of |substmingoth|.
\item |boldbyembolden| (default)\Means
  Changes the implementation of |\CJKbold| (pseudo-bold) from
  ``overstriking'' to ``synthetic emboldening''.
\item |noboldbyembolden|\Means
  Negation of |boldbyembolden|.
\item |CJKtildeasspace|\Means
  When \Pkg{hyperref} is used, |~| in text will be converted
  to a space character in generating PDF strings.
  This is the same behavior as when this package is not used.
\item |noCJKtildeasspace| (default)\Means
  Negation of |CJKtildeasspace|, that is, |~| in text
  will be deleted in PDF strings.
\end{itemize}

%===========================================================
\section{Usage}
\label{sec:usage}

%-------------------
\subsection{Selecting CJK fonts}

The present package provides three ``generic'' CJK families in the same
way as {\pLaTeX} plus the
\href{http://www.ctan.org/pkg/japanese-otf}{\Pkg{japanese-otf} package}:
Mincho family (|\mcfamily|), Gothic family (|\gtfamily|), and
Maru-gothic family (|\mgfamily|).
In default setting, the font set from the \Pkg{ipaex-type1} package
are allocated;
Mincho family uses IPAex Mincho font, and Gothic and Maru-gothic
families use IPAex Gothic font.
This allocation can be altered by users.

\begin{itemize}
\item |\mcfamily|\Means
  Changes the CJK family to Mincho family.\\
  Equivalent to |\CJKfamily{\mcdefault}|.
\item |\gtfamily|\Means
  Changes the CJK family to Gothic family.\\
  Equivalent to |\CJKfamily{\gtdefault}|.
\item |\mgfamily|\Means
  Changes the CJK family to Maru-gothic family.\\
  Equivalent to |\CJKfamily{\mgdefault}|.
\end{itemize}

More advanced commands:

\begin{itemize}
\item |\mcdefault|/|\gtdefault|/|\mgdefault|\Means
  The names of CJK families corresponding to the three generic families.
  In the standard allocation their values are
  |mc|/\wbr|gt|/\wbr|mg| respectively
  and the allocation is used as default.
\item
  |\setCJKfamilydefault{|\Meta{CJK-family}|}|\Means
  Declares the default CJK family.
  This default value is used when family names are missing in some
  commands, such as |\CJKfamily{}| and
  |\begin{CJK}{UTF8}{}|.
  The (redefined) |\normalfont| also switches the CJK family to the
  family specified by this command.

  The default value of this default family is the ``counterpart'' of the
  alphabetic font family which is in effect at the beginning of the
  document body.
  (See the next section.)
\end{itemize}

%-------------------
\subsection{Synchronization of CJK and non-CJK families}

The \Pkg{CJK} package (and {\pTeX} engine) manages separate
``current families'' for CJK and alphabetic (non-CJK) families.
While this treatment has its merit, synchronization of the two
``current families'' is convenient in many cases.
Accordingly, the present package redefines some of the {\LaTeX}
commands that switches current alphabetic font families so that
the CJK family will be switched to the counterpart of the current
alphabetic family, where the ``counterpart'' is defined as follows:

\begin{itemize}
\item |\rmfamily| (Serif) → |\mcfamily| (Mincho)
\item |\sffamily| (Sans-serif) → |\gtfamily| (Gothic)
\item |\ttfamily| (Monospace) → |\gtfamily| (Gothic)
\item The counterpart of the other families is |\mcfamily|.
\end{itemize}

Redefined commands:

\begin{itemize}
\item |\rmfamily|/|\sffamily|/|\ttfamily|\Means
  Changes the CJK family to the counterpart of the alphabetic font
  family after executing the original function.
\item |\normalfont|\Means
  Changes the CJK family to the default CJK family that is specified
  by the |\setCJKfamilydefault| command.
\end{itemize}

There are shorthand forms of |CJK|/\wbr|CJK*| environments:

\begin{itemize}
\item |\begin{uCJK*}|\ldots|\end{uCJK*}|\Means
  Equivalent to:
\begin{verbatim}
\begin{CJK*}{UTF8}{counterpart}...\end{CJK*}
\end{verbatim}
  where |counterpart| means the counterpart of the current alphabetic
  font family.

  Note that this is \emph{not} equivalent to
\begin{verbatim}
\begin{CJK*}{UTF8}{}...\end{CJK*}
\end{verbatim}
  structure, which uses the default CJK family.
\item |\begin{uCJK}|\ldots|\end{uCJK}|\Means
  Equivalent to:
\begin{verbatim}
\begin{CJK}{UTF8}{counterpart}...\end{CJK}
\end{verbatim}
\end{itemize}

%-------------------
\subsection{Font mapping}

The usage of these commands are the same as in the \Pkg{pxchfon}
package.
Please refer to the manual of that package for detail.

\begin{itemize}
\item |\setminchofont{|\Meta{id}|]{|\Meta{font-file}|}|
\item |\setgothicfont{|\Meta{id}|]{|\Meta{font-file}|}|
\item |\setmarugothicfont{|\Meta{id}|]{|\Meta{font-file}|}|
\item |\setmediumminchofont{|\Meta{id}|]{|\Meta{font-file}|}|
\item |\setboldminchofont{|\Meta{id}|]{|\Meta{font-file}|}|
\item |\setmediumgothicfont{|\Meta{id}|]{|\Meta{font-file}|}|
\item |\setboldgothicfont{|\Meta{id}|]{|\Meta{font-file}|}|
\item |\setxboldgothicfont{|\Meta{id}|]{|\Meta{font-file}|}|
\end{itemize}

However there is a major limitation as to the use of font mapping with
the {\pdfTeX} engine.
One can use only TrueType fonts and moreover TTC format is not allowed.
(One can use any flavor of OpenType fonts when using dvipdfmx.)

\Note The present package does not support the light-weight Mincho
font, and thus the |\setlightminchofont| command does nothing useful.

%-------------------
\subsection{Other commands}

\begin{itemize}
\item |\UTF{|\Meta{hexadecimal-number}|}|\Means
  Inputs a CJK character through Unicode codepoint value.
  |\UTF{5B57}| is equivalent to |\Unicode{"5B}{"57}|.
\item |\CJKforce{|\Meta{character}\ldots|}|\Means
  Afterwards treats the characters given in the argument as CJK
  characters (printed using CJK fonts).
\item |\CJKunforce{|\Meta{character}\ldots|}|\Means
  Cancels the effect of the |\CJKforce| command.
\item |\@|\Meta{character}\Means
  Treats the next character (only that occurrence) as a CJK character,
  when the character is outside the ASCII range; othersize the
  standard meaning of |\@| is retained.
\item |\CJKecglue|\Means
  Inserts a ``shibuaki'' space.
  This will be invoked by |~| when |\CJKtilde| is in effect.
  This command can be redefined by users to adjust the value of
  shibuaki space, just as |\CJKglue| can be redefined to adjust
  inter-ideographic space.

  For example:
\begin{verbatim}
\renewcommand{\CJKecglue}{\hspace{0.125em minus 0.125em}}
\end{verbatim}
\end{itemize}

%===========================================================
\section{Remarks}
\label{sec:remarks}

\begin{itemize}
\item The standard font families provided by this package does
  \emph{not} support vertical writing, even when using default
  ipaex-type1 font set.
  However, the families provided by \Pkg{ipaex-type1}
  (|ipxm| and |ipxg|) do support vertical writing, and one can
  utilize these families directly by specifying |ipaex-type1| option.
\end{itemize}

%===========================================================
\end{document}
