% This file is encoded in UTF-8.
% To be typeset with pdflatex or latex + dvipdfmx.
\documentclass[12pt]{beamer}           % for pdfTeX
%\documentclass[12pt,dvipdfmx]{beamer} % for dvipdfmx
\hypersetup{unicode}%← needed to make bookmark text right
\usepackage[whole]{bxcjkjatype}% needed 'whole'
%% That'all! You can do 日本語 right now!
\usetheme{Warsaw}
%---------------------------------------
\title{How to do 日本語 with pdf{\TeX}}
\author{ZR 某}
\date{某日 of August, 2013}
%---------------------------------------
\begin{document}

\begin{frame}
  \titlepage
\end{frame}

%---------------------------------------
\section{日本語 with bxcjkjatype package}

\begin{frame}[fragile]{First}
  Load the \alert{bxcjkjatype} package in preamble.\par
  Use of \structure{\texttt{whole}} option is often suitable.\par
  \begin{exampleblock}{とある Example}
\begin{verbatim}
\documentclass[a4paper]{article}
\usepackage[whole]{bxcjkjatype}
\end{verbatim}
  \end{exampleblock}
\end{frame}

\begin{frame}[fragile]{Second}
  Write whatever 日本語 text you like in the document body.\par
  You can use ひらがな, カタカナ, 漢字, and/or
  any character available in the standard fonts
  (\structure{IPAex明朝/ゴシック}).\par
  \begin{exampleblock}{とある Example}
\begin{verbatim}
\documentclass[a4paper]{article}
\usepackage[whole]{bxcjkjatype}
\begin{document}
私のホバークラフトは鰻でいっぱいです。
\end{document}
\end{verbatim}
  \end{exampleblock}
  \structure{※} Must use UTF-8.\par
\end{frame}

\begin{frame}[fragile]{Third}
  Compile the document as usual.\par

\begin{verbatim}
pdflatex sample.tex
\end{verbatim}
\end{frame}

\begin{frame}{Last}\relax
  {\LARGE\alert{完成!}\par}
  \begin{block}{とある Output}
    \rmfamily
    私のホバークラフトは鰻でいっぱいです。
  \end{block}
\end{frame}

%---------------------------------------
\section{Conclusion (Not!)}

\begin{frame}{おしまい}
  \transdissolve[duration=0.5]
  \centering\Large
  \structure{Happy 日本語 {\TeX}ing!}
\end{frame}

\end{document}
