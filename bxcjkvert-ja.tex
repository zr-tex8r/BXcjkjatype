% LuaLaTeX 文書; 文字コードは UTF-8
\documentclass[a4paper]{ltjsarticle}
\usepackage{geometry}
\usepackage{xcolor}
\usepackage[unicode,colorlinks,hyperfootnotes=false]{hyperref}
\hypersetup{linkcolor=blue!75!black,urlcolor=green!45!black}
\usepackage{shortvrb}
\MakeShortVerb{\|}
\usepackage{verbatim}
\newenvironment{myverbatim}
  {\begin{quote}\small\verbatim}
  {\endverbatim\end{quote}}
\newcommand{\PkgVersion}{0.4}
\newcommand{\PkgDate}{2016/11/11}
\newcommand{\Pkg}[1]{\textsf{#1}}
\newcommand{\Meta}[1]{$\langle$\mbox{}#1\mbox{}$\rangle$}
\newcommand{\Note}{\par\noindent ※}
\newcommand{\Means}{:\quad}
\newcommand{\jemph}{\textsf}
\newcommand{\wbr}{\linebreak[0]}
\newcommand{\Cs}[1]{\texttt{\symbol{`\\}#1}}
%-----------------------------------------------------------
\begin{document}
\title{\Pkg{bxcjkvert} パッケージ}
\author{八登崇之\ (Takayuki YATO; aka.~``ZR'')}
\date{v\PkgVersion\quad[\PkgDate]}
\maketitle

\begin{abstract}
本パッケージは\Pkg{CJKvert}パッケージの改造版であり、
縦組と横組の混在が普通に起こりうる日本語組版に適合させたものである。
\end{abstract}

\tableofcontents

%===========================================================
\section{パッケージ読込}
\label{sec:loading}

\begin{verbatim}
\usepackage[<option>,...]{bxcjkvert}
\end{verbatim}

本パッケージは\Pkg{CJKvert}パッケージを内部で読み込む。

利用可能なオプションを以下で挙げる。

%-------------------
\subsection{フォントリセットに関する設定}

\Pkg{CJKvert}パッケージは、書字方向変更の命令
(|\CJKhorz| および |\CJKvert|)の呼出の際に、
|\normalfont| を実行して現在フォントをリセットする。
この挙動は異なる書字方向を混在させる場合には特に不便である。
従って、\Pkg{bxcjkvert}は既定ではこの挙動を抑止している。
ただし |resetfont| オプションによりこの挙動は調整可能である。

\begin{itemize}
\item |resetfont=true|\Means
  書字方向命令が現在フォントをリセット(|\normalfont|)する。
  \Pkg{CJKvert}ではこれが既定である。
\item |resetfont=false|(既定)\Means
  書字方向命令が現在フォントを変更しない。
\end{itemize}

%-------------------
\subsection{ベースライン伸長の補正に関する設定}

\Pkg{CJKvert}は書字方向変更時に |\baselinestretch| の値に対する
補正を行う。
具体的には、|\CJKvert| が有効な間は
ベースライン伸長値が |\CJKbaselinestretch| 倍
\footnote{\Cs{CJKbaselinestretch}の既定値は$1.3$である。}
に増大する。
ところが、日本語組版に関する限り、ベースライン伸長値を
変えるべき理由はどこにもない。
従って、\Pkg{bxcjkvert}はこの機能を無効化している。
ただし |usebaselinestretch| オプションによりこの挙動は
調整可能である。


\begin{itemize}
\item |usebaselinestretch=true|\Means
  |usebaselinestretch| オプション付きの\Pkg{CJKvert}パッケージと同じ。
  すなわち、|\CJKvert| 実行時にはベースライン伸長値
  を |\CJKbaselinestretch| 倍にし、|\CJKhorz| 実行時には
  ベースライン伸長値を復元する。%
  \footnote{ただし、この場合、ベースライン伸長値は
  「\Pkg{CJKvert}パッケージ読込時に保持されていた値」に戻される
  ことに注意。}
\item |usebaselinestretch=false|\Means
  |usebaselinestretch| オプション無しの\Pkg{CJKvert}パッケージと同じ。
  すなわち、|\CJKvert| 実行時にはベースライン伸長値
  を |\CJKbaselinestretch| の値に設定し、|\CJKhorz| 実行時には
  ベースライン伸長値を$1$に設定する。
  (ユーザによるベースライン伸長値の設定は無視される。)
\item |usebaselinestretch=retain|(既定)\Means
  ベースライン伸長値に対する変更を一切行わない。
\end{itemize}

\Note \Pkg{CJKvert}が予め |usebaselinestretch| オプション付きで
読み込まれていた場合は、本パッケージの |usebaselinestretch| の
既定値は(|retain| ではなく)|true| になる。

%-------------------
\subsection{書字方向の初期値に関する設定}

\Pkg{CJKvert}では初期(文書開始時)の書字方向は縦組と
定められている。
本パッケージでは初期の書字方向をユーザが選択することができる。

\begin{itemize}
\item |main=true|\Means
  書字方向の初期値を縦組とする。
  \Pkg{CJKvert}ではこれが既定である。
\item |main=false|\Means
  書字方向の初期値を横組とする。
\item |main=retain| (default)\Means
  書字方向の初期値を何も指定しない。
  この場合、プレアンブルで |\CJKvert| や |\CJKhorz| を実行することで
  書字方向の初期値を決めることができる。
\end{itemize}

%===========================================================
\end{document}
