% LuaLaTeX 文書; 文字コードは UTF-8
\documentclass[a4paper]{ltjsarticle}
\renewcommand{\headfont}{\gtfamily\romanseries{sbc}\sffamily}
\usepackage{geometry}
\usepackage{xcolor}
\usepackage[unicode,colorlinks,hyperfootnotes=false]{hyperref}
\hypersetup{linkcolor=blue!75!black,urlcolor=green!45!black}
\usepackage{shortvrb}
\MakeShortVerb{\|}
\usepackage{verbatim}
\newenvironment{myverbatim}
  {\begin{quote}\small\verbatim}
  {\endverbatim\end{quote}}
\newcommand{\PkgVersion}{0.5}
\newcommand{\PkgDate}{2023/07/23}
\newcommand{\Pkg}[1]{\textsf{#1}}
\newcommand{\Meta}[1]{$\langle$\mbox{}#1\mbox{}$\rangle$}
\newcommand{\Note}{\par\noindent ※}
\newcommand{\Means}{:\quad}
\newcommand{\Strong}{\textsf}
\newcommand{\wbr}{\linebreak[0]}
\providecommand{\pTeX}{p\TeX}
\providecommand{\pLaTeX}{p\LaTeX}
\providecommand{\pdfTeX}{pdf\/\TeX}
%-----------------------------------------------------------
\begin{document}
\title{\Pkg{bxcjkjatype} パッケージ}
\author{八登崇之\ (Takayuki YATO; aka.~``ZR'')}
\date{v\PkgVersion\quad[\PkgDate]}
\maketitle

\begin{abstract}
本パッケージは、日本語組版に適した\Pkg{CJK}パッケージの設定を提供する。
その上で、{\pLaTeX}ユーザにとって\Pkg{CJK}パッケージの使用が容易に
するために、{\pLaTeX}カーネルや{\pLaTeX}用パッケージのものに
類似した命令を提供する。
\end{abstract}

\tableofcontents

%===========================================================
\section{パッケージ読込}
\label{sec:loading}

\begin{verbatim}
\usepackage[オプション,...]{bxcjkjatype}
\end{verbatim}

利用可能なオプションを以下で挙げる。

%-------------------
\subsection{CJK環境の自動付加}

文書本体を |CJK(*)| 環境で自動的かつ安全に囲うためのオプション。
文書全体においてCJKを有効にしたい場合、
および「動く引数」にCJKテキストが含まれる場合は
これを使うのが適切である。

\begin{itemize}
\item |whole|、|wholeCJK*|\Means
  文書全体を |CJK*| 環境で囲う。
  \Note 厳密に言うと |\begin{uCJK*}~\end{uCJK*}| で囲う。
\item |wholeCJK|\Means
  文書全体を |CJK| 環境で囲う。
  \Note 厳密に言うと |\begin{uCJK}~\end{uCJK}| で囲う。
\item |nowhole|(既定)\Means
  |wholeCJK*| 及び |wholeCJK| の否定。
\end{itemize}

%-------------------
\subsection{「自動チルダ」}

{\pTeX}エンジンや\Pkg{luatexja}パッケージと異なり、
\Pkg{CJK}パッケージには和欧文間空白(いわゆる“四分空き”)を
自動挿入する機能はないので、
文書作成者が手動で入れる必要がある。
この負担を軽減するため、\Pkg{CJK}パッケージでは
「非分割欧文空白を入れる命令であるチルダ“|~|”を
和欧文間空白を入れる命令に切り替える」
ための機構を用意している。
|\CJKtilde| という命令を実行すると“|~|”が和欧文間空白に切り替わる%
\footnote{|\standardtilde| は |\CJKtilde| の効力を打ち消して
“|~|”の意味を本来の非分割欧文空白に戻す。
また、|\CJKtilde| 有効時に非分割欧文空白を入れるために、
常に非分割欧文空白を表す |\nbs| という命令が用意されている。}。

本パッケージで |autotilde| オプションを指定すると、
|CJK(*)| 環境に入ったときに |\CJKtilde| が自動的に呼び出される
ようになる。
このため |CJK(*)| 環境内では“|~|”が和欧文間空白の意味になる。

\begin{itemize}
\item |autotilde|\Means
  全ての |CJK(*)| 環境の先頭で |\autotilde| が呼び出されるようにする。
\item |noautotilde|(既定)\Means
  |autotilde| の否定。
\end{itemize}

%-------------------
\subsection{PDF文字列中の和欧文間空白の扱い}

先述の通り、{\LaTeX}の文法ではチルダ“|~|”は非分割欧文空白を表す。
従って、\Pkg{hyperref}パッケージがPDF文字列を生成する際には、
{\LaTeX}テキスト中の“|~|”は空白文字に変換される。

ところが |\CJKtilde| が有効である場合はこの挙動には問題がある。
|\CJKtilde| 適用時は“|~|”の意味が和欧文間空白に変わるが、
和欧文間空白は組版上の現象でありテキスト情報としての空白ではない
からである。
従って、本パッケージを読み込んだ場合には、
|\CJKtilde| 適用時の“|~|”はPDF文字列への変換の際には
削除される仕様に変更される。

さらに、この挙動はオプションで制御することができる。

\begin{itemize}
\item |noCJKtildeasspace|(既定)\Means
  |\CJKtilde| 適用時の“|~|”はPDF文字列では削除される。
\item |CJKtildeasspace|\Means
  |\CJKtilde| 適用時の“|~|”はPDF文字列では空白文字に
  変換される。
  \Note 本来の\Pkg{hyperref}の挙動と同じ。
\end{itemize}

%-------------------
\subsection{フォントマップ設定}

\href{http://www.ctan.org/pkg/pxchfon}{\Pkg{pxchfon}パッケージ}%
と同等のフォントマップのプリセットが利用できる。
詳細についてはそちらの解説文書を参照されたい。

\begin{itemize}
\item |oneweight|、|nooneweight|\Means
  \Pkg{pxchfon}と同様。
\item \Pkg{pxchfon}で利用可能なプリセットオプション(|ms| 等)が
  利用できる。
  (非推奨のものを除く。)
\item |ttfname=|\Meta{パターン}\Means
  TTC代替(\ref{ssec:ttfname}節)を行う場合のTTFフォント名のパターン。
\item |ipaex-type1|\Means
  本パッケージのフォント管理を無効にし、\Pkg{ipaex-type1}パッケージ
  のファミリ(|ipxm| と |ipxg|)を直接用いる。
  この設定では |\mcdefault| の値は |ipxm| に、|\gtdefault| の
  値は |ipxg| になる。
\end{itemize}

%-------------------
\subsection{CJKフォントスケール}

\begin{itemize}
\item |scale=|\Meta{実数}\Means
  CJKフォントに対するスケール値を設定する。
\end{itemize}

\Note 0.3版より、|ipaex-type1| 指定時もCJKフォントスケールが
利用できるようになった。

%-------------------
\subsection{その他のオプション}

\begin{itemize}
\item |everypage|\Means
  フォントマップ情報を出力DVI文書の全てのページに出力する。
  |dvipdfmx| ドライバでのみ有効。
\item |noeverypage|(既定)\Means
  |everypage| の否定。
\item ドライバオプション\Means
  |pdftex|、|dvipdfmx|、|dvips|、|none| が指定できる。
  ドライバ設定は既定(ipaex-type1)以外のフォントマップの使用時に
  のみ意味をもつ。
  さらに、フォントマップ変更は |pdftex| と |dvipdfmx| でのみ
  サポートされ、またこの2つの値は常に自動判定可能
  (PDFモードでは |pdftex|、DVIモードでは |dvipdfmx| が既定)
  なので、実際にはドライバを指定する必要はない。
\item |substmingoth|\Means
  CJKパッケージの日本語用総称ファミリとして慣習的に使われている
  ファミリの |min|/|goth|/|maru| について、{\LaTeX} レベルで、
  本パッケージにおける標準ファミリである |mc|/|gt|/|mg| への
  代替を設定する。
  これにより、|min| 等のファミリの使用を前提とした日本語用
  テンプレート等を使う場合においても、論理フォントレベルでの代替設定を
  行う必要がなくなる。
\item |nosubstmingoth|(既定)\Means
  |substmingoth| の否定。
\item |boldbyembolden|(既定)\Means
  \Pkg{CJK}パッケージの“擬似太字”(|\CJKbold|)の処理方法を
  “重ね打ち”から“合成太字”に変更する。
\item |noboldbyembolden|\Means
  |boldbyembolden| の否定。
\end{itemize}

%-------------------
\subsection{TTC代替}
\label{ssec:ttfname}

{\pdfTeX}エンジンはTTC形式のフォントをサポートしていない
(サポートしているOpenTypeの形式は単体のTTFグリフのものだけである)。
従って、|\setminchofont| 命令等でTTCフォントを使おうと望むなら、
適当なツールを使ってTTCフォントファイルを個々のTTFフォントファイル
に分解する処理を行うしかない\footnote{%
もちろん、ライセンスの規定のためそれができない場合もある。}。

これはエンジン側の制限仕様のため仕方がないだろう。
しかし本パッケージを用いる場合には他に問題がある。
変換後のTTFフォントはファイル名が元のものと異なるため、
プリセット設定(|mobo-moga| 等のオプション)が使えないのである。
これを解決するために、0.3版で |ttfname| オプションが新設された。

例えば、パッケージ読込時に |ttfname=*_1| が指定されたとする。
この状態で本パッケージの処理において「mogam.ttcの最初(0番)」
が要求された場合、代わりに「mogam\_1.ttf」というファイルが使われる。
同様に、「mogam.ttcの1番、2番、……」については
「mogam\_2.ttf」「mogam\_3.ttf」……での代替が行われる。

|ttfname| で指定するファイル名のパターンには「|*_1|」
「|TEMP-*-00.TTF|」のように「|*| を1つ含み、数字列を含む文字列」
が指定できる。
この |*| の部分はプリセットが要求するTTCファイルの
ベース名(|mogam| 等)で
置き換えられ、かつ、拡張子がない場合は |.ttf| が補われる。
その結果の文字列がTTCの最初(0番)のフォントに対する
代替ファイル名となる。
後続(1番、2番、……)の代替ファイル名は数字部分を
インクリメントしたものになる。

%===========================================================
\section{機能}
\label{sec:usage}

%-------------------
\subsection{CJKフォントの選択}

本パッケージでは、{\pLaTeX} +
\href{http://www.ctan.org/pkg/japanese-otf}{\Pkg{japanese-otf}パッケージ}%
で用いられるものと同等の3つの「総称」CJKフォントファミリを用意している:
明朝(|\mcfamily|)、ゴシック(|\gtfamily|)、
丸ゴシック(|\mgfamily|)。
既定では\Pkg{ipaex-type1}パッケージのフォントが割り当てられている:
明朝→IPAex明朝、ゴシック・丸ゴシック→IPAexゴシック。
この割当は変更可能である。

\begin{itemize}
\item |\mcfamily|\Means
  CJKファミリを明朝ファミリに変更する。
  \Note |\CJKfamily{mcdefault}| と等価である。
\item |\gtfamily|\Means
  CJKファミリをゴシックファミリに変更する。
  \Note |\CJKfamily{gtdefault}| と等価である。
\item |\mgfamily|\Means
  CJKファミリを丸ゴシックファミリに変更する。
  \Note |\CJKfamily{mgdefault}| と等価である。
\end{itemize}

高度な命令\Means

\begin{itemize}
\item |\mcdefault|/|\gtdefault|/|\mgdefault|\Means
  総称ファミリに対応するCJKファミリ名である。
  標準の割当ではこれらの値は |mc|/|gt|/|mg| でありこの割当が
  既定で用いられる。
\item |\setCJKfamilydefault{|\Meta{CJKファミリ}|}|\Means
  既定のCJKファミリを設定する。
  この既定値は、ある種の命令(例えば |\CJKfamily{}|
  や |\begin{CJK}{UTF8}{}|)でファミリ名が省略された時に用いられる。
  (再定義後の)|\normalfont| はCJKファミリをこの命令で設定した
  既定ファミリに変更する。

  この既定ファミリの既定値は、文書の本体の先頭で有効になっている
  欧文ファミリの「対応ファミリ」
  (\ref{ssec:sync-families}節)となる。
\end{itemize}

%-------------------
\subsection{CJKと欧文のファミリの連動}
\label{ssec:sync-families}

\Pkg{CJK}パッケージ(および{\pTeX}エンジン)ではCJKと欧文で
別々の「現在ファミリ」を管理する。
この取扱は利点もあるが、多くの場合はこの2つの「現在ファミリ」を
連動させた方が都合がよい。
このため、本パッケージでは現在欧文フォントを切り替える幾つかの
{\LaTeX}命令について、CJKファミリを欧文の「対応ファミリ」に
切り替えるように改修する。
ここで「対応ファミリ」は以下のように定められる:

\begin{itemize}
\item |\rmfamily|(セリフ) → |\mcfamily|(明朝)
\item |\sffamily|(サンセリフ) → |\gtfamily|(ゴシック)
\item |\ttfamily|(等幅) → |\gtfamily|(ゴシック)
\item その他のファミリについては、対応ファミリは |\mcfamily| とする。
\end{itemize}

改修される命令:

\begin{itemize}
\item |\rmfamily|/|\sffamily|/|\ttfamily|\Means
  本来の動作の後、CJKファミリを欧文ファミリの「対応ファミリ」
  に変更する。
\item
  |\normalfont|\Means
  CJKファミリを、|\setCJKfamilydefault| 命令により指定された
  既定CJKファミリに変更する。
\end{itemize}

|CJK|/|CJK*| 環境の省略系\Means

\begin{itemize}
\item |\begin{uCJK*}...\end{uCJK*}|\Means
  次のものと等価\Means
\begin{verbatim}
\begin{CJK*}{UTF8}{対応ファミリ}...\end{CJK*}
\end{verbatim}
  ただし“|対応ファミリ|”は現在の欧文ファミリの対応ファミリを表す。

  次のものと等価では\Strong{ない}ことに注意\Means
\begin{verbatim}
\begin{CJK*}{UTF8}{}...\end{CJK*}
\end{verbatim}
  こちらは既定のCJKファミリ(|\setCJKfamilydefault| の設定)
  を使用する。

\item |\begin{uCJK}...\end{uCJK}|\Means
  次のものと等価\Means
\begin{verbatim}
\begin{CJK}{UTF8}{対応ファミリ}...\end{CJK}
\end{verbatim}
\end{itemize}

%-------------------
\subsection{フォントマップ設定}

これらの命令の使い方は\Pkg{pxchfon}パッケージの時と同じであるので、
詳細については\Pkg{pxchfon}の解説文書を参照されたい。

\begin{itemize}
\item |\setminchofont[|\Meta{ID}|]{|\Meta{フォントファイル名}|}|
\item |\setgothicfont[|\Meta{ID}|]{|\Meta{フォントファイル名}|}|
\item |\setmarugothicfont[|\Meta{ID}|]{|\Meta{フォントファイル名}|}|
\item |\setmediumminchofont[|\Meta{ID}|]{|\Meta{フォントファイル名}|}|
\item |\setboldminchofont[|\Meta{ID}|]{|\Meta{フォントファイル名}|}|
\item |\setmediumgothicfont[|\Meta{ID}|]{|\Meta{フォントファイル名}|}|
\item |\setboldgothicfont[|\Meta{ID}|]{|\Meta{フォントファイル名}|}|
\item |\setxboldgothicfont[|\Meta{ID}|]{|\Meta{フォントファイル名}|}|
\end{itemize}

ただし、{\pdfTeX}エンジン使用時は、
エンジン自体の制限仕様のため、
「TrueTypeグリフであってかつTTC形式で\Strong{ない}フォント」
のみが使用できる。
(dvipdfmx使用時は全ての種類のOpenTypeフォントが使用可能。)

\Note 
本パッケージは「明朝・細字」のフォントをサポートしない。
従って、|\setlightminchofont| は動作しない。

%-------------------
\subsection{その他の命令}

\begin{itemize}
\item |\UTF{|\Meta{16進数字}|}|\Means
  CJK文字をUnicode符号値で入力する。
  例えば、|\UTF{5B57}| は |\Unicode{"5B}{"57}| と等価である。
\item |\CJKforce{|\Meta{文字}…|}|\Means
  以降は引数の中にある各々の文字をCJK文字として扱う
  (CJKフォントで出力される)。
\item
  |\CJKunforce{|\Meta{文字}…|}|\Means
  |\CJKforce| の効果を打ち消す。
\item |\@|\Meta{文字}\Means
  非ASCII文字の前に |\@| がある場合は、その文字(その出現のみ)を
  CJK文字として扱う。
  ASCII文字の前の |\@| は{\LaTeX}本来の意味を維持する。
\item |\CJKecglue|\Means
  和欧文間空白を挿入する。
  |\CJKtilde| が有効時の“|~|”に相当する
  (“|~|”実際にこの命令を呼び出す)。
  この命令をユーザが再定義して和欧文間空白の量を調節することが
  できる(和文間空白の |\CJKglue| と同様)。

  使用例\Means
\begin{verbatim}
\renewcommand{\CJKecglue}{\hspace{0.125em minus 0.125em}}
\end{verbatim}
\end{itemize}

%===========================================================
\section{補足事項}
\label{sec:remarks}

\begin{itemize}
\item 本パッケージで標準に用いられるフォントは、例えそれに
  ipaex-type1のフォントが割り当てられている場合でも、縦書きを
  サポートしない。
  しかし\Pkg{ipaex-type1}が提供するファミリ(|ipxm| と |ipxg|)そのものは
  縦書きをサポートしており、|ipaex-type1| オプションを指定することで
  それらのファミリを直接使うことができる。
\end{itemize}

%===========================================================
\end{document}
